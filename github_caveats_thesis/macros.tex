%      $Id: macros.tex 506 2009-10-05 16:57:07Z daniel $    

\usepackage{booktabs}
\usepackage{relsize}
\usepackage{xspace}
\usepackage{subfigure}
\usepackage{listings}
%\lstloadlanguages{java}
\DeclareGraphicsRule{*}{pdf}{*}{}
\newcommand{\otoprule}{\midrule[\heavyrulewidth]}
\newcommand{\pldi}{ACM Programming Language Design and Implementation (PLDI)}
\newcommand{\taco}{ACM Transactions on Architecture and Code Optimization (TACO)}
\newcommand{\lctes}{ACM Languages, Compiler, and Tool Support for Embedded Systems (LCTES)}
\newcommand{\popl}{ACM Principles of Programming Languages (POPL)}
\newcommand{\ecoop}{European Conference for Object-Oriented Programming (ECOOP)}
\newcommand{\asplos}{ACM Architectural Support for Programming Languages and Operating Systems (ASPLOS)}
\newcommand{\sigmetrics}{ACM Measurement and Modeling of Computer Systems (SIGMETRICS)}
\newcommand{\oopsla}{ACM Object-Oriented Programming, Systems, Languages, and Applications (OOPSLA)}
\newcommand{\ismm}{International Symposium on Memory Management (ISMM)}
\newcommand{\veee}{ACM/USENIX Virtual Execution Environments (VEE)}
\newcommand{\micro}{ACM/IEEE International Symposium on Microarchitecture}
\newcommand{\isca}{ACM/IEEE International Symposium on Computer Architecture (ISCA)}
\newcommand{\icse}{International Conference  on Software Engineering (ICSE)}
\newcommand{\pact}{Parallel Architectures and Compilation Techniques (PACT)}
\newcommand{\casess}{ACM Compilers, Architectures, and Synthesis for Embedded Systems (CASES)}

\definecolor{tableheadcolor}{rgb}{0.8,0.8,1.0}
%\definecolor{tablealtcolor}{rgb}{0.9,0.9,1.0}
\definecolor{tablealtcolor}{rgb}{0.9,0.9,0.95}


\definecolor{todocolor}{rgb}{0.8,0.8,1.0}
\definecolor{fixcolor}{rgb}{1,0.8,0.8}
\definecolor{commentcolor}{rgb}{0.8,1.0,0.8}


\newcommand{\listingfigure}[3]{
\begin{figure}[ht!]
  \begin{center}
    \begin{minipage}[t]{\textwidth-4cm}
      \lstinputlisting{#1}
    \end{minipage}
  \end{center}
  \caption{#3}#2
\end{figure}}

\newcommand{\includeabchart}[5]{
\begin{figure}[ht!]
\begin{center}
\newcommand{\atitle}{#4}
\newcommand{\btitle}{#5}
\input{charts/#1.tex}
\end{center}
\caption{#3}#2
\end{figure}}

\newcommand{\placeholderfigure}[2]{
\begin{figure}[ht!]
  \begin{center}
    \resizebox{\textwidth-2cm}{0.7\textwidth-1.4cm}{todo}
  \end{center}
  \caption{#2}#1
\end{figure}}

\newcommand{\singlegraphfigure}[3]{
\begin{figure}[ht!]
  \begin{center}
    \includegraphics[width=\textwidth-2cm]{#1}
  \end{center}
  \caption{#3}#2
\end{figure}}

\usepackage[color=todocolor, colorinlistoftodos]{todonotes}

%\newcommand{\notinpart}{%
% \def\toclevel@chapter{-1}\def\toclevel@section{0}\def\toclevel@subsection{1}} \newcommand{\inpart}{
% \def\toclevel@chapter{0}\def\toclevel@section{1}\def\toclevel@subsection{2}}


%
% Stuff for pretty printing the source code using listings.sty
%


%% set Java as the default language
\lstset{
  numbers=left,
  numberstyle=\tiny,
  stepnumber=1,
  numbersep=2em,
  language=java,                         % the language
  basicstyle=\footnotesize\ttfamily,     % the basic font family to use
  commentstyle=\itshape,                 % the font for comments
  stringstyle=\ttfamily,
%  morekeywords={@Intrinsic, @Unboxed, @RawStorage}
}
%\lstset{language=java}

\newcommand{\textjava}[1]{{\lstset{basicstyle=\ttfamily}\lstinline@#1@}}
\newcommand{\textjavafn}[1]{{\lstset{basicstyle=\footnotesize\ttfamily}\lstinline@#1@}}
%\usepackage{lstasm}
\usepackage{setspace}
\usepackage{ifthen}
%\usepackage{color}
%\usepackage{smallheadings}

\long\def\sfootnote[#1]#2{\begingroup%
\def\thefootnote{\fnsymbol{footnote}}\footnote[#1]{#2}\endgroup}
%
% code
%

\newcommand{\address}{\textjava{Address}\xspace}
\newcommand{\ubregion}{\textjava{unbump-region()}\xspace}
\newcommand{\word}{\textjava{Word}\xspace}
\newcommand{\freeme}{\textjava{free()}\xspace}
\newcommand{\freemeunbump}{\textjava{unbump()}\xspace}
\newcommand{\freemeunbumpregion}{\textjava{unbump-region()}\xspace}
\newcommand{\freemeunreserve}{\textjava{unreserve()}\xspace}

%
% abbreviations
%


\newcommand{\eg}{e.g., }
\newcommand{\ie}{i.e., }

\newcommand{\GenMS}{\emph{GenMS}\xspace}
\newcommand{\GenImmix}{\emph{GenIX}\xspace}
\newcommand{\mmtk}{MMTk\xspace}
\newcommand{\jikes}{Jikes RVM\xspace} 
\newcommand{\jikesrvm}{\jikes} 
\newcommand{\jala}{Jalape\~{n}o\xspace} 
\newcommand{\jalapeno}{Jalape\~{n}o\xspace} 

\newcommand{\dacapo}{\textsf{DaCapo}\xspace}
\newcommand{\specjvm}{\textsf{SPECjvm98}\xspace}
\newcommand{\cattrack}{\textsf{cattrack}\xspace}
\newcommand{\spec}{\textsf{SPEC}\xspace}

\newcommand{\nurserytype}[1]{{\fontfamily{cmss}\selectfont \textsl{#1}}}
\newcommand{\alloc}{\nurserytype{allocate}\xspace}
\newcommand{\collect}{\nurserytype{collect}\xspace}
\newcommand{\redirect}{\nurserytype{redirect}\xspace}

\newcommand{\bmtype}[1]{{\textsf{#1}}}

\newcommand{\jbb}{\bmtype{jbb2000}\xspace}
\newcommand{\psjbb}{\bmtype{pjbb2005}\xspace}
\newcommand{\pjbb}{\bmtype{pjbb2005}\xspace}
\newcommand{\specjbb}{\bmtype{SPECjbb2005}\xspace}
\newcommand{\jess}{\bmtype{jess}\xspace}
\newcommand{\raytrace}{\bmtype{raytrace}\xspace}
\newcommand{\db}{\bmtype{db}\xspace}
\newcommand{\javac}{\bmtype{javac}\xspace}
\newcommand{\jack}{\bmtype{jack}\xspace}
\newcommand{\compress}{\bmtype{compress}\xspace}
\newcommand{\mpegaudio}{\bmtype{mpegaudio}\xspace}
\newcommand{\mtrt}{\bmtype{mtrt}\xspace}
\newcommand{\antlr}{\bmtype{antlr}\xspace}
\newcommand{\bloat}{\bmtype{bloat}\xspace}
\newcommand{\chart}{\bmtype{chart}\xspace}
\newcommand{\eclipse}{\bmtype{eclipse}\xspace}
\newcommand{\fop}{\bmtype{fop}\xspace}
\newcommand{\hsqldb}{\bmtype{hsqldb}\xspace}
\newcommand{\jython}{\bmtype{jython}\xspace}
\newcommand{\luindex}{\bmtype{luindex}\xspace}
\newcommand{\lusearch}{\bmtype{lusearch}\xspace}
\newcommand{\Lusearch}{\bmtype{Lusearch}\xspace}
\newcommand{\pmd}{\bmtype{pmd}\xspace}
\newcommand{\ps}{\bmtype{ps}\xspace}
\newcommand{\SPECjbb}{\bmtype{SPECjbb}\xspace}
\newcommand{\xalan}{\bmtype{xalan}\xspace}
\newcommand{\sunflow}{\bmtype{sunflow}\xspace}
\newcommand{\Sunflow}{\bmtype{Sunflow}\xspace}
\newcommand{\avrora}{\bmtype{avrora}\xspace}
\newcommand{\core}{Core2 Quad\xspace}
\newcommand{\corelong}{Intel Core2 Quad Q6600\xspace}
\newcommand{\phenom}{Phenom II\xspace}
\newcommand{\phenomlong}{AMD Phenom II X6 1055T\xspace}
\newcommand{\sandy}{i7-2600\xspace}
\newcommand{\sandylong}{Intel Core i7-2600\xspace}



\newcommand{\ghostscript}{\bmtype{ghostscript}\xspace}

\newcommand{\doi}[1]{\href{http://dx.doi.org/#1}{\nolinkurl{doi:#1}}}
%
% misc
%
\newcommand{\fix}[1]{\todo[color=fixcolor]{#1}}
\newcommand{\comment}[1]{\todo[color=commentcolor]{#1}}
\newcommand{\ifix}[1]{\todo[inline,color=fixcolor]{#1}}
\newcommand{\icomment}[1]{\todo[inline,color=commentcolor]{#1}}
\newcommand{\itodo}[1]{\todo[inline]{#1}}
\newcommand{\ignore}[1]{}
\newcommand{\mccenter}[1]{\multicolumn{1}{c|}{#1}}

%
% figure spacing
%
%\clubpenalty 10000
%\widowpenalty 10000
%\def\topfraction{0.9}
%\def\bottomfraction{0.9}
%\def\textfraction{0.1}
%\renewcommand{\singlespacing}{\renewcommand{\baselinestretch}{1.00}\small\normalsize}
%\renewcommand{\doublespacing}{\renewcommand{\baselinestretch}{1.5}\small\normalsize}
%\newcommand{\tight}{\renewcommand{\baselinestretch}{1.28}\small\normalsize}
%\renewcommand{\subfigbottomskip}{0.25ex}
%\renewcommand{\subfigcapskip}{0ex}
%\renewcommand{\subfigcapskip}{-1ex}
%\newcommand{\subfigshrink}{-0.75ex}
%\newcommand{\subfigcapspace}{2ex}

%\newcommand{\subwidth}[0]{.32\textwidth}


%
% margins
%
%\topmargin -.5truein
%\textheight 9truein
%\oddsidemargin .25truein
%\evensidemargin .25truein
%\textwidth 6truein


%
% crossreferencing footnotes
%
%\newcommand{\fnref}[1]{~(\ref{#1})}
%\newcommand{\onecolparbox}{3.1in}


%\newcommand{\textjava}[1]{{\lstset{language=java,basicstyle=\footnotesize\ttfamily}\lstinline@#1@}}
%\newcommand{\textasm}[1]{{\lstset{language=asm,basicstyle=\footnotesize\ttfamily}\lstinline@#1@}}

%%
%% Change the sections etc.
%%
%\makeatletter
%\parskip=0pt
%\renewcommand\section{\@startsection{section}{1}{\z@}%
%                                   {-2.5ex}% beforeskip
%%                                   {1ex}% afterskip
%                                   {\large \bfseries \raggedright}}
% \renewcommand\subsection{\@startsection{subsection}{2}{\z@}%
%                                     {-2ex\@plus -1ex \@minus -.2ex}%
%                                      {.5ex \@plus .2ex}%
%                                      {\normalsize \bfseries \raggedright}}
% \renewcommand\subsubsection{\@startsection{subsubsection}{3}{\z@}%
%                                      {-2ex\@plus -1ex \@minus -.2ex}%
%                                      {1ex \@plus .2ex}%
%                                      {\normalfont\fontsize{11pt}{12pt}\selectfont\itshape}}
%\renewcommand{\thesubsubsection}{\thesubsection.\arabic{subsubsection}}

%\renewcommand\paragraph{\@startsection{paragraph}{4}{\z@}% 
%  {.5em}%
%  {-1em}%
%  {\normalfont\normalsize\bfseries\parskip=0pt}}
%\setlength\partopsep{0\p@}
%\setlength\parskip{0\p@ \@plus \p@}

%\makeatother
%\parindent=9pt





%%% Local Variables: 
%%% mode: latex
%%% TeX-master: "doa"
%%% End:

\usepackage{xcolor}

\definecolor{pblue}{rgb}{0.13,0.13,1}
\definecolor{pgreen}{rgb}{0,0.5,0}
\definecolor{pred}{rgb}{0.9,0,0}
\definecolor{pgrey}{rgb}{0.46,0.45,0.48}
\definecolor{gray}{rgb}{0.4,0.4,0.4}
\definecolor{darkblue}{rgb}{0.0,0.0,0.6}
\definecolor{cyan}{rgb}{0.0,0.6,0.6}


\lstdefinelanguage{None}{}

\usepackage{array}
\newcolumntype{C}[1]{>{\centering\arraybackslash}p{#1}}
\usepackage{multirow}
% verbatim lstlisting
\lstset{
	language=None,
	basicstyle=\ttfamily,
	columns=fixed,
	fontadjust=true,
	basewidth=0.5em
}

\lstset{language=Java,
	basicstyle=\ttfamily\footnotesize,
	showspaces=false,
	showtabs=false,
	breaklines=true,
	showstringspaces=false,
	breakatwhitespace=true,
	commentstyle=\color{pgreen},
	keywordstyle=\color{pblue},
	stringstyle=\color{pred},
	basicstyle=\ttfamily,
	literate={\ \ }{{\ }}1,
	numbers=left,
	stepnumber=1,
}

\usepackage{makecell}
\renewcommand\theadalign{bc}
\renewcommand\theadfont{\bfseries}
\renewcommand\theadgape{\Gape[4pt]}
\renewcommand\cellgape{\Gape[4pt]}

\usepackage{hyperref}
\usepackage{amsmath}
\usepackage{tabularx}
\usepackage{lstlinebgrd}

\usepackage{expl3,xparse}

\ExplSyntaxOn
\NewDocumentCommand \lstcolorlines { O{pink} m }
{
	\clist_if_in:nVT { #2 } { \the\value{lstnumber} }{ \color{#1} }
}
\ExplSyntaxOff