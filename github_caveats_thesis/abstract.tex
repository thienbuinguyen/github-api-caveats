\chapter*{Abstract}
\addcontentsline{toc}{chapter}{Abstract}
\vspace{-1em}
This thesis investigates the methods of linking Application Programming Interface (API) caveats to code. API caveats deal with the accessibility of API reference documentation and are defined as the constraints that specify what users are allowed or not allowed to do. Previous research has suggested that API caveats contribute a significant amount to the issues developers face when using an API. Linking API caveats to code can, therefore, help developers learn about correct API usage. A previously proposed method applied word2vec sentence embedding and other Natural Language Processing (NLP) techniques to API caveat sentences and community text from Stack Overflow answers. Cosine similarity was used to determine the similarity of the API caveat and community text vectors, allowing sample code from community text to be indirectly linked to API caveats. This thesis applies the previous approach to GitHub data to determine its effectiveness when applied to a different domain. I extract 73,831 API caveat sentences from the Java 12 API documentation, and 629,933 sentences from GitHub comment sentences related to Java repositories throughout 2018. I then create 4 information retrieval systems based on TF-IDF, word2vec, BM25 and word2vec + BM25 to link the caveat sentences with GitHub issue comment sentences. However, a significant lexical gap was discovered given that only 1\% of query results were relevant to the query API caveat for all information retrieval systems tested. In comparison to previous work, the main issue was concluded to be the different purposes of GitHub text data: bug reporting and suggestions instead of general Q\&A. Furthermore, GitHub data is considerably noisier and can be relevant to a multitude of APIs while Stack Overflow provides a clear distinction between different question/post categories for data analysis. The negative results prompted an investigation into a more direct method to link API caveats to code. The concept of \textit{caveat contracts} is introduced, which are similar to code contracts that can detect bugs while coding, but are derived from API caveat sentences. I use and combine proposed methods of sentence normalisation and heuristics to parse the Java 12 API caveat sentences and extract \textit{not-null} and \textit{range limitation} constraints. This is used to construct 4,694 unique contracts for the Java 12 API. An evaluation of this approach shows an accuracy, precision, recall and F1 score of 0.77, 0.96, 0.73, and 0.83 respectively. I then develop an IntelliJ \textit{checker} plugin that utilises these caveat contracts and static code analysis to automatically check for API misuse in real-time. \\
Overall, this thesis presents the complete process of linking API caveats to code via the use of caveat contracts. The key contributions include a simplistic parsing approach to extract constraints from a subset of exception related API caveats and the implementation of a checker program that serves as a proof-of-concept for real-time bug detection.