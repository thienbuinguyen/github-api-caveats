\chapter{Introduction}
\label{cha:intro}
An Application Programming Interface (API) provides a set of functions to interact with some software. However, the correct usage of APIs often requires knowledge and understanding specific to that API. Incorrect usage, which is referred to as API misuse in this thesis, can lead to severe bugs for developers. Fortunately, API reference documentation typically provides information to guide developers to correct usage by describing the constraints associated with different components of the API. A recent study by \cite{caveat-knowledge-graph} researched a particular subset of these constraints that describe what developers can and cannot do. This class of constraints is termed as \textit{API caveats}. An approach for extracting these API caveats was also proposed based on using their syntactical patterns. Improving the accessibility of API documentation by helping developers understand API caveats was then investigated by \cite{jiamou}. The approach used NLP techniques with sentence embedding of Android API caveat sentences and community text from the Q/A forum Stack Overflow, which was used to indirectly collect code examples from the programming community and augment API documentation. This thesis seeks to extend upon the approach in \cite{jiamou} by testing it in a different domain abundant with open source code: GitHub. Besides this, alternative methods of linking API caveats to code are investigated to improve comprehension of API caveats in general. In particular, I focus on constructing \textit{caveat contracts} from the natural language of API caveats. This concept is similar to code contracts and can be used in static code analysis tools to check for API misuses in real-time. \\

\section{Motivation}
Suppose a developer was starting their journey on learning Java. A topic that may pique their interest is file handling: how to read text files into their programs. A common first strategy involves using Google to search for ``java how to read files'' (for example). The first result of this is a tutorial webpage by GeeksforGeeks\footnote{https://www.geeksforgeeks.org/different-ways-reading-text-file-java/} that provides multiple methods and examples for reading a file in Java. The first example is shown in Listing \ref{lst:code-example}.

\centering
\begin{lstlisting}[tabsize=4,caption={File reading java code example from GeeksforGeeks},label={lst:code-example},numbers=left]
// Java Program to illustrate reading from FileReader 
// using BufferedReader 
import java.io.*; 
public class ReadFromFile2 
{ 
	public static void main(String[] args)throws Exception 
	{ 
		// We need to provide file path as the parameter: 
		// double backquote is to avoid compiler interpret words 
		// like \test as \t (ie. as a escape sequence) 
		File file = new File("C:\\Users\\pankaj\\Desktop\\test.txt"); 
		
		BufferedReader br = new BufferedReader(new FileReader(file)); 
		
		String st; 
		while ((st = br.readLine()) != null) 
		System.out.println(st); 
	} 
} 
\end{lstlisting}
\flushleft

At this point, the next step a developer might take is to copy the code shown in Listing \ref{lst:code-example} directly into their program. Pleased with their clever thinking, the developer might try to execute their program only to find a \lstinline{FileNotFoundException} is thrown. This is because the file path in the \lstinline{File} constructor call has not been changed to the appropriate file path relative to their system. In an Integrated Development Environment (IDE) such as IntelliJ, information about the exception would also be exposed. This includes the associated line number in which the exception was thrown. However, a confounding result would be discovered: the exception is not associated with the \lstinline{File} constructor in line 11 (where the file path is set), but with the \lstinline{FileReader} constructor from line 13. In other words, the file path appears to be accepted by \lstinline{File}. \bigbreak

A logical next step would involve searching for the reference documentation of these Java classes\footnote{https://docs.oracle.com/javase/7/docs/api/java/io/FileReader.html \\\indent https://docs.oracle.com/javase/7/docs/api/java/io/File.html}. By reading the documentation of \lstinline{FileReader}'s constructor, the developer will discover a line that says ``Throws: FileNotFoundException - if the file does not exist, is a directory rather than a regular file, or for some other reason cannot be opened for reading''. Also checking the documentation of \lstinline{File}'s constructor, they will find that no such warning is provided. Rather, they may notice that \lstinline{File} contains a \lstinline{exists()} method used to verify whether a file exists \textit{after} instantiating the \lstinline{File} object. Only after this arduous process does the developer understand the source of the problem (the file path specified in this scenario) alongside correct usage of \lstinline{File} and \lstinline{FileReader} for reading files.\\
Although the above example presents a simplistic view of how a new developer might approach learning an API, we observe the constraints associated with an API introduce a significant problem for all programmers. This problem lies in the fact that ``you don't know what you don't know'', meaning the usage of an API requires a considerable understanding of the different components and their interactions before they can be used. This is a time-consuming endeavour given the number and complexity of different APIs.\bigbreak

For reference, several examples of caveats from the Java 12 API are shown in Tables \ref{tab:caveat-eg2}, \ref{tab:caveat-eg3} and \ref{tab:caveat-eg1}. Note that the community text provided is from answers or comments addressing an issue raised by another developer involving the API caveat. The questions/posts generally include superfluous information and are excluded for brevity. We observe from \ref{tab:caveat-eg2} and \ref{tab:caveat-eg3} that community solutions for API caveat related problems essentially attempt to describe the API caveats in a simplified manner. Table \ref{tab:caveat-eg1} showcases an example where the API documentation does not mention the caveat, but community answers provide the missing information. Hence, it can be seen that API caveats provide essential information for correct API usage, but its portrayal/accessibility may be insufficient due to issues such as complex descriptions or lack of examples. A previously proposed solution was to augment API documentation by using text and code examples from the programming community, which was mainly focused on Q\&A platforms such as Stack Overflow.  However, other community-centered platforms like GitHub exist that contain significantly more code given its use for hosting Free and Open-Source Software, but lack the community text structure of a Q\&A website. This thesis, therefore, attempts to test whether previously applied approaches could obtain better results in a different domain (GitHub) and whether other methods of linkage can be used to connect API caveats to code.
\clearpage
\begin{table}[h]
	\centering
	\begin{tabular}{|c|c|}
		\hline
		& \textbf{API Caveat} \\ \hline
		\textbf{API Element} & \textit{String.indexOf(int n)} \\ \hline
		\textbf{\begin{tabular}[c]{@{}c@{}}Method Summary\end{tabular}} & \begin{tabular}[c]{@{}c@{}}Returns the index of the first \\ occurrence of the specified character \\ in this string\end{tabular} \\ \hline
		\textbf{Caveat Description} & \begin{tabular}[c]{@{}c@{}}Indices are positive, but this method can return a -1 value\end{tabular} \\ \hline
		\textbf{\begin{tabular}[c]{@{}c@{}}Relevant \\ Stack Overflow \\ Comment\\ \#1\end{tabular}} & \begin{tabular}[c]{@{}c@{}}``If <code>indexOf</code> cannot find the required \\ string, it returns <code>-1</code>.''\footnotemark \\ \end{tabular} \\ \hline
		\textbf{\begin{tabular}[c]{@{}c@{}}Relevant \\ Stack Overflow \\ Comment\\ \#2\end{tabular}} & \begin{tabular}[c]{@{}c@{}}``If \textless{}code\textgreater{}.\textless{}/code\textgreater  isn't present, \\ <code>indexOf</code> returns -1 as a sentinel''\footnotemark \\ \end{tabular} \\ \hline
	\end{tabular}
	\caption{Example of a caveat for \lstinline{String.indexOf(int n)} alongside relevant Stack Overflow/GitHub comments.}
	\label{tab:caveat-eg2}
\end{table}
\footnotetext[3]{https://stackoverflow.com/questions/40314998/java-substring-out-of-range-1}
\footnotetext{https://stackoverflow.com/questions/10859633/java-string-index-out-of-range-1-exception-using-indexof}

\begin{table}[h]
	\centering
	\begin{tabular}{|c|c|}
		\hline
		& \textbf{API Caveat} \\ \hline
		\textbf{API Element} & \textit{ArrayList.subList(int fromIndex, int toIndex)} \\ \hline
		\textbf{\begin{tabular}[c]{@{}c@{}}Method Summary\end{tabular}} & \begin{tabular}[c]{@{}c@{}}Returns a view of the portion of this \\ list between the specified fromIndex, \\ inclusive, and toIndex, exclusive\end{tabular} \\ \hline
		\textbf{Caveat Description} & \begin{tabular}[c]{@{}c@{}}the starting index (fromIndex) is \\ inclusive while the ending index (toIndex)\\ is exclusive\end{tabular} \\ \hline
		\textbf{\begin{tabular}[c]{@{}c@{}}Relevant \\ GitHub \\ Comment\end{tabular}} & \begin{tabular}[c]{@{}c@{}}``...subList toIndex is exclusive...''\footnotemark\\ \\ \end{tabular} \\ \hline
		\textbf{\begin{tabular}[c]{@{}c@{}}Relevant \\ Stack Overflow \\ Comment\end{tabular}} & \begin{tabular}[c]{@{}c@{}}``The first index is inclusive, the other is exclusive.''\footnotemark\end{tabular} \\ \hline
	\end{tabular}
	\caption{Example of a caveat for \lstinline{ArrayList.subList(int fromIndex, int toIndex)} alongside relevant Stack Overflow/GitHub comments.}
	\label{tab:caveat-eg3}
\end{table}
\footnotetext[5]{https://github.com/viritin/viritin/issues/273}
\footnotetext{https://stackoverflow.com/questions/37256605/why-does-arraylist-sublist0-n-return-a-list-of-size-n}
\clearpage
\begin{table}[h]
	\centering
	\begin{tabular}{|c|c|}
		\hline
		& \textbf{API Caveat} \\ \hline
		\textbf{API Element} & \textit{HashMap.size()} \\ \hline
		\textbf{\begin{tabular}[c]{@{}c@{}}Method Summary\end{tabular}} & \begin{tabular}[c]{@{}c@{}}Returns the number of key-value \\ mappings in this map\end{tabular} \\ \hline
		\textbf{Caveat Description} & \begin{tabular}[c]{@{}c@{}}Negative values can be returned in\\ multi-threaded environments\end{tabular} \\ \hline
		\textbf{\begin{tabular}[c]{@{}c@{}}Relevant \\ GitHub \\ Comment\end{tabular}} & \begin{tabular}[c]{@{}c@{}}``Since HashMap is not threadsafe it's \\ possible to get HashMap size as -1 \\ if you try removing items concurrently.''\footnotemark \\ \end{tabular} \\ \hline
		\textbf{\begin{tabular}[c]{@{}c@{}}Relevant \\ Stack Overflow \\ Comment\end{tabular}} & \begin{tabular}[c]{@{}c@{}}``HashMap is not thread-safe. Therefore, \\ if there is any point where two threads \\ could update a HashMap without \\ proper synchronization, the map \\ could get into an inconsistent state.''\footnotemark \\ \end{tabular} \\ \hline
	\end{tabular}
	\caption{Example of a caveat for \lstinline{HashMap.size()} alongside relevant Stack Overflow/GitHub comments.}
	\label{tab:caveat-eg1}
\end{table}
\footnotetext[7]{https://github.com/RuedigerMoeller/fast-serialization/issues/223}
\footnotetext{https://stackoverflow.com/questions/5283294/can-a-java-hashmaps-size-be-out-of-sync-with-its-actual-entries-size}

\section{Main Research Challenges}
\label{sec:mainresearchchallenges}
The primary goal of this thesis is the investigation of methods for linking API caveats to code. Ideally, this would allow the detection of API misuse and help developers understand the correct usage of a given API. Previous work has focused on an indirect approach of performing this linkage: caveats sentences were matched with community text surrounding code examples on Q\&A websites such as Stack Overflow. This utilised NLP techniques involving sentence embedding to form information retrieval systems. The caveat sentences were then used as queries and the information retrieval systems used to locate similar community text with associated code examples from Stack Overflow. For this thesis, one objective is to determine whether the same approach could be applied to a different domain such as GitHub. This is a complex task that involves four components: (1) extraction of API caveats, (2) sentence embedding, (3) matching caveats sentences to similar sentences from GitHub, and (4) presenting code examples to developers. The scope of this project covers components 2 and 3. This is because the extraction method used by \cite{jiamou} was found to be sufficient for allowing sentences to be matched (component 1). Furthermore, a simple Graphical User Interface (GUI) was found to help users better understand API caveats (component 4). In terms of component 2, sentence embedding is a modern approach in NLP that transforms sentences into vectors. These vectors retain the semantic meaning of their original sentences \cite{palangi2016deep}. However, this typically requires complex models (such as neural networks) to learn important features of sentences. \bigbreak

Matching API caveats to code examples is also non-trivial given the semantic/lexical gap that exists between natural language and programming code. An example can be seen with the API caveat sentence ``UnsupportedEncodingException - If the named charset is not supported'' from the Java 12 Documentation (method \lstinline{getBytes} of class \lstinline{String}). Developing a generalised representation of this for computers to understand is particularly difficult. It requires consideration of context, that ``charset'' refers to a parameter named ``charsetName'', alongside what exactly denotes a supported ``charset''. The fourth component refers to the problem of how to present API documentation in an understandable way. Overall, these components only represent a part of the challenges of this thesis. As negative results were found using the previous approach for GitHub, an alternative method of linking API caveats to code was investigated. This introduced additional research challenges involving sentence parsing and static code analysis.\bigbreak

Sentence parsing involves numerous difficulties in the process of interpreting semantic meaning. It requires identification of individual words, contextual information, punctuation and subtle features such as intonation \cite{mitchell1994sentence}. Several fields of NLP exist to determine specific properties of a sentence, such as Named Entity Recognition for identifying which entity certain words refer to. An example of why sentence parsing is a complex topic is provided by \cite{ratinov-roth-2009-design} with the news headline ``SOCCER - PER BLINKER BAN LIFTED''. Without some background knowledge, it is difficult to recognize that ``BLINKER'' refers to a soccer player.  Therefore, the scope of this thesis in terms of sentence parsing is limited to applying and extending existing parsing techniques to construct caveat contracts. \bigbreak

Other challenges for this thesis are those involved with static code analysis, which refers to the process of examining the source code of programs before executing them \cite{baca2009static}. This is typically used to ``find bugs and reduce defects in a software application'' \cite{bardas2010static}. However, implementing static code analysers, which we refer to as \textit{checkers}, is a non-trivial task. This is observed with existing static analysis tools that can detect complex vulnerabilities but also commonly produce false-positive alarms \cite{zitser2004testing}. Besides this, the performance of checker programs must also be considered as increasing the complexity of checks would also require more difficult computations and more time. The scope of this thesis is for static code analysis is limited to implementing a checker program that can utilise caveat contracts. 
\clearpage
\section{Thesis Outline}
\label{sec:outline}
An overview of API caveats and the background work of this thesis is described in Chapter \ref{cha:background}. Chapter \ref{cha:infoRetrieval} extends upon the work of \cite{jiamou} and \cite{xiaoxue} for linking API caveat sentences to code examples in a different domain: GitHub. A more direct approach for linkage is described in Chapter \ref{cha:codeAnalysis} with the construction of caveat contracts and static code analysis. In Chapter \ref{cha:conc}, the findings of this thesis are discussed with remarks for future work.

\section{Main Contributions}
The main contributions of Chapter \ref{cha:infoRetrieval} is the discovery of a significant lexical gap between community text from GitHub and (Java 12) API caveat sentences, which impedes previous approaches used for linking API caveats to code examples.\\

\noindent
The main contributions of Chapter \ref{cha:codeAnalysis} are:
\begin{itemize}
	\item Parsing of a subset of API caveats from the Java 12 API to formulate caveat contracts.
	\item Implementation of a checker program (IntelliJ plugin) that uses static code analysis to automatically check caveat contract compliance in real time.
\end{itemize}