\chapter{Related Work}
\label{cha:background}
%At the begging of each chapter, please introduce the motivation and high-level
%picture of the chapter. You also have to introduce sections in the
%chapter. \\
%
%Section~\ref{sec:relatedwork} yyyy.\\

\section{Related work}
\label{sec:relatedwork}

\section{API Documentation and API Caveats}
\begin{enumerate}
	\item \cite{jiamou} - Jiamou report
	\item \cite{caveat-knowledge-graph} - Knowledge graph
	\item \cite{xiaoxue} - trustdoc
	\item \cite{construct-knowledge-graph} - knowledge graph thesis
	\item \cite{uddin2015api} - API doc study
\end{enumerate}

\section{NLP focus for API}
\begin{enumerate}
\item \cite{live-api-doc} - BAKER - link source code to API documentation
\item \cite{zheng2017code} - Code to comments
\item \cite{van2017combining} - using w2v on code for finding code examples
\item \cite{silva2019recommending} - CROKAGE
\item \cite{li2018api} - API Caveat Explorer
\end{enumerate}

\section{API Usage Misuse}
\begin{enumerate}
	\item \cite{code-examples} - API call sequence mining on GitHub
	\item \cite{zhou-directive} - Directive defects
	\item \cite{mutation-analysis} - Mutation analysis
	\item \cite{amann2016mubench} - API misuse detector benchmark MuBENCH
	\item \cite{kapur2018estimating} - defectiveness of code on GitHub
	\item \cite{bae2014safewapi} -API misuse detection on web apps
\end{enumerate}

\section{Summary}



