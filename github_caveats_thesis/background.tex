\chapter{Related Work}
\label{cha:background}
This chapter provides an overview of the background for API caveats, linking API caveats to code examples with NLP techniques, and the use of static code analysis to detect API errors or misuses. \\
The term \textit{API reference documentation} or \textit{API documentation} is adopted throughout this thesis to refer to the set of \textit{documents} that are indexed by an \textit{API element} such as a class or method (\cite{maalej2013patterns}). For example, the Java 12 reference documentation consists of documents as web-pages with each describing a specific Java class (\lstinline{String}, \lstinline{ArrayList} etc.). \\

Section \ref{sec:related-api-caveats} references the papers that termed \textit{API caveats} and extended upon this concept for linkage of API caveats to code.\\

Section \ref{sec:related-nlp} mentions related works that have applied NLP techniques to the domain of API documentation.\\

Section \ref{sec:related-static-code-analysis} references related works that applied static code analysis to the domain of API documentation.\\

\section{API Caveats}
\label{sec:related-api-caveats}
API reference documentation consists of a taxonomy of knowledge types that are defined by \cite{maalej2013patterns}. A particular knowledge type identified as \textit{directives} ``specifies what users are allowed/not allowed to do with the API element''. This is identified as a notable component of API reference documentation by \cite{caveat-knowledge-graph} that includes all forms of constraints and is therefore referred to as \textit{API caveats}. These constraints are of particular interest as they can be used to determine how accessible an API document is. Specifically, accessibility of documentation is an essential part of any API/framework because it is used to describe functionality and usage of the associated API. \citeauthor{caveat-knowledge-graph} conducted a formative study of the Q\&A website Stack Overflow to highlight the prevalence of issues involving API caveats for programmers. Also, a set of syntactic patterns were identified that could be used to recognise API caveats (see Table \ref{tab:caveat-keywords}). Moreover, API caveats have many practical applications such as the examination of the quality/validity of Stack Overflow answers \cite{xiaoxue}, construction of a knowledge graph for information retrieval purposes or entity-centric searches of API caveats \cite{construct-knowledge-graph}, and augmentation of caveats with code examples \cite{jiamou}. In particular, the work by \cite{jiamou} is the foundations for Chapter \ref{cha:infoRetrieval}. It was shown that one possible solution for linking caveats to code is by indirectly using community text from Q\&A websites. This approach used the NLP techniques of mainly word2vec sentence embedding to the sentences of API caveats and answer posts on Stack Overflow. The cosine similarity of these vector outputs could then be used to determine the relatedness of these vectors, which represents sentence similarity. The code examples from the answer and question posts could then be inferred as ``good'' and ``bad'' code examples respectively for the given API caveat. 

\section{Natural Language Processing for API Documents}
\label{sec:related-nlp}
A code-analysis centered solution for the linkage of API documents and code is proposed by \cite{live-api-doc}. This approach uses deductive linking, where the Abstract Syntax Tree (AST) of a particular code snippet from online sites is analysed, the relevant API elements are detected, and then linked to associated API documents. Another solution combines NLP and code analysis to detect errors such as obsolete code samples \cite{zhong2013detecting}. For this, the names of API elements from code samples represented with natural language or as code is compared to the set of all API element names for some API. Mismatches in the names are then reported as documentation errors. Another interesting approach for detecting API misuse does not consider any API documentation and data-mines the most common API usage patterns from a large set of repositories on GitHub \cite{code-examples}. This is accomplished by traversing the AST of code samples to generate API call sequences that represent important information about a single API call alongside relevant expressions/statements surrounding it. The correct usage patterns are then applied to Stack Overflow to determine how reliable code examples are on Q\&A platforms. Besides this, \cite{zhou-directive} provides a method of detecting defects between API documents and their code implementations.  The constraints from directives and code are extracted as FOL expressions, which are then compared using a Satisfiability Modulo Theories (SMT) solver. In particular, this work identifies several categories of constraints for directives that represent the largest portion of API documentation at 43.7\% \cite{monperrus2012should}. Heuristic patterns and regular expressions used from this work are the building blocks for constructing caveat contracts in Chapter \ref{cha:codeAnalysis}.
The concept of mutation analysis has also been suggested for exposing API misuse \cite{mutation-analysis}. Mutation analysis involves many, small modifications to a program called \textit{mutants} that are then executed with a given test suite. Execution data from the stack trace of these programs can then be used to determine whether a mutant introduced an API misuse and recognise patterns misuse. Finally, an example of the application of static code analysis for detecting bugs is shown by \cite{bae2014safewapi} for JavaScript web applications. We note that JavaScript is a dynamically typed programming language in comparison to Java, which is a statically typed language. However, static type analysis can still be used to perform type-based analysis with some modifications. Overall, the method proposed uses the fact that Web APIs are typically specified in a certain format known as Interface Definition Language, which exposes function semantics.

\section{Summary}
This chapter provides an overview of the related works for API caveats, NLP applied to API documentation, and the use of static code analysis for API misuse detection. In particular, key background information for the work in Chapters \ref{cha:infoRetrieval} and \ref{cha:codeAnalysis} is provided, and some key terminology used throughout this thesis is defined to avoid ambiguity for readers. In Chapter \ref{cha:infoRetrieval}, the techniques in \cite{jiamou} are applied to GitHub data to investigate linking API caveats to a different community platform domain (GitHub).


